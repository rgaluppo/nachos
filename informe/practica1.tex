% ------ Clase de documento ------
\documentclass[a4paper, 11pt]{article}

% ----------- Preámbulo ----------

% Paquetes
\usepackage[T1]{fontenc}      % Codificación de salida
\usepackage[spanish]{babel}
\usepackage{color}
\usepackage[utf8]{inputenc}
\usepackage{mathptmx}         % Selección de la fuente (tipo de letra)

\usepackage{color}
\definecolor{gray97}{gray}{.97}
\definecolor{gray75}{gray}{.75}
\definecolor{gray45}{gray}{.45}

\usepackage{listings}
\lstset{ frame=Ltb,
framerule=0pt,
aboveskip=0.5cm,
framextopmargin=3pt,
framexbottommargin=3pt,
framexleftmargin=0.4cm,
framesep=0pt,
rulesep=.4pt,
backgroundcolor=\color{gray97},
rulesepcolor=\color{black},
%
stringstyle=\ttfamily,
showstringspaces = false,
basicstyle=\small\ttfamily,
commentstyle=\color{gray45},
keywordstyle=\bfseries,
%
numbers=left,
numbersep=15pt,
numberstyle=\tiny,
numberfirstline = false,
breaklines=true,
}

% minimizar fragmentado de listados
\lstnewenvironment{listing}[1][]
{\lstset{#1}\pagebreak[0]}{\pagebreak[0]}

\lstdefinestyle{consola}
{basicstyle=\scriptsize\bf\ttfamily,
backgroundcolor=\color{gray75},
}

\lstdefinestyle{C}
{language=C++,
}

% Configuración (para artículo)
\title{Resolución Práctica 1 de Nachos}
\author{Galuppo, Raúl I. Graf, Andrea}
\date{\today}

% ----- Cuerpo del documento -----
\begin{document}
\maketitle

1.  La máquina simulada tiene 32 páginas de 128 bytes.

2.  Para cambiar ese valor hay que modificar dos valores:
            + la cantidad de páginas.
            + el tamaño de una página.
    El primero se encuentra en code/machine/machine.h :
        + NumPhysPages. Actualmente su valor es 32.

    El segundo se encuentr en code/machine/disk.h :
        + SectorSize. Actualmente su valor es 128.


 A continuacion se listan:
ADD – Add (with overflow)
ADDI -- Add immediate (with overflow)
ADDIU -- Add immediate unsigned (no overflow)
ADDU -- Add unsigned (no overflow)
AND -- Bitwise and
ANDI -- Bitwise and immediate
BEQ -- Branch on equal
BGEZ -- Branch on greater than or equal to zero
BGEZAL -- Branch on greater than or equal to zero and link
BGTZ -- Branch on greater than zero
BLEZ -- Branch on less than or equal to zero
BLTZ -- Branch on less than zero
BLTZAL -- Branch on less than zero and link
BNE -- Branch on not equal
DIV -- Divide
DIVU -- Divide unsigned
J -- Jump
JAL -- Jump and link
JALR -- Jump and link register
JR -- Jump register
LB -- Load byte
LBU -- Load byte unsigned
LH -- Load half word
LHU -- Load half word unsigned
LUI -- Load upper immediate
LW -- Load word
LWL -- Load word left
LWR -- Load word right
MFHI -- Move from HI
MFLO -- Move from LO
MTHI -- Move to hi
MTLO -- Move to lo
MULT -- Multiply
MULTU -- Multiply unsigned
NOR -- Put the logical NOR of the integer
OR -- Bitwise or
ORI -- Bitwise or immediate
RFE -- Return from exception
SB -- Store byte
SH -- Store half word 
SLL -- Shift left logical
SLLV -- Shift left logical variable
SLT -- Set on less than (signed)
SLTI -- Set on less than immediate (signed)
SLTIU -- Set on less than immediate unsigned
SLTU -- Set on less than unsigned
SRA -- Shift right arithmetic
SRAV -- Shift right arithmetic variable
SRL -- Shift right logical
SRLV -- Shift right logical variable
SUB -- Subtract
SUBU -- Subtract unsigned
SW -- Store word
SWL -- Store word left
SWR -- Store word right
SYSCALL -- System call
XOR -- Bitwise exclusive
XORI -- Bitwise

5.

+ registers:  es un arreglo que contiene los valores que contienen los registros. Sus indices son los registros de mips.
+ inst: Definida en /code/machine/machine.h

\begin{lstlisting}[style=C]
class Instruction {
  public:
    void Decode();	// decode the binary representation of the instruction

    unsigned int value; // binary representation of the instruction

    char opCode;     // Type of instruction.  This is NOT the same as the
    		     // opcode field from the instruction: see defs in mips.h
    char rs, rt, rd; // Three registers from instruction.
    int extra;       // Immediate or target or shamt field or offset.
                     // Immediates are sign-extended.
};
\end{lstlisting}

En la implementación de la instruccion ADD se utilizan inst->rs (nombre del primer registro fuente ) y instr->rt (nombre del segundo registro), instr->rd (nombre del registro destino).

$+ Constante SIGN_BIT: definida en /code/machine/mipssim.h
Es una mascara cuyo valor es  0x80000000. Nos sirve para saber que signo tiene el nro.
$
\begin{lstlisting}[style=C]
   case OP_ADD:
	sum = registers[(int)instr->rs] + registers[(int)instr->rt];
	if (!((registers[(int)instr->rs] ^ registers[(int)instr->rt]) & SIGN_BIT) &&
	    ((registers[(int)instr->rs] ^ sum) & SIGN_BIT)) {
	    RaiseException(OverflowException, 0);
	    return;
	}
	registers[(int)instr->rd] = sum;
	break;
\end{lstlisting}

Overflow en una suma solo ocurre cuando sumo dos números del mismo signo y el resultado tiene distinto signo. Esto esuc porque el acarreo cambia el bit de signo.
Entonces hacemos que la condición del if sea:      \begin{verbatim}~ ( sign1 XOR sign2) && (sign1 XOR signResult)\end{verbatim}
Estos valores lo calculamos de la siguiente manera:
\begin{verbatim}
    sign1 = num1 & SIGN_BIT
    sign2 = num2 & SIGN_BIT
    signResult = sum & SIGN_BIT
\end{verbatim}

Reference: Calculating Overflow Flag: Method 1 - http://teaching.idallen.com/dat2343/10f/notes/040_overflow.txt

6. 
    el metodo DEBUG se encuentra en /code/threads/utility.cc
    ASSERT /code/threads/utility.h  
    el metodo Initialize se encuentra en /code/threads/system.cc
    el metodo ThreadTest se encuenta en /code/threads/threadtest.cc
    StartProcess() /code/userprog/progtest.cc
    ConsoleTest() /code/userprog/progtest.cc
    Halt() /code/machine/interrupt.cc pero es un metodo definido en la clase Interrupt, la cual se encuentra en /code/machine/interrupt.h
    Copy() /code/filesys/fstest.cc
    Print() /code/filesys/fstest.cc 
    Remove()/code/filesys/directory.cc
    List()/code/filesys/directory.cc
    Print() /code/filesys/directory.cc //Print sin argumentos. 
    PerformanceTest() /code/filesys/fstest.cc
    Delay() /code/machine/sysdep.cc
    MailTest() /code/network/nettest.cc
    Finish() /code/threads/thread.cc

7.  Se necesitaria una maquina con MIPS para correr Nachos o que Nachos implemente otro conjunto de instrucciones (Intel 64, etc).
    Usando la maquina real, si por error un test genera un proceso que cae en deadlock, habria que reiniciar la maquina real. Al simular la CPU, podemos matar ese proceso y volver a testear.
    Lo mismo sucederia si un test escribe mal en el disco o en la RAM (Segmentation Fault, por ejemplo).

8. 	The debugging routines allow the user to turn on selected debugging messages, controllable from the command line arguments passed to Nachos (-d).  You are encouraged to add your own debugging flags.  The pre-defined debugging flags are:

        '+' -- turn on all debug messages
        't' -- thread system
        's' -- semaphores, locks, and conditions
        'i' -- interrupt emulation
        'm' -- machine emulation (USER_PROGRAM)
        'd' -- disk emulation (FILESYS)
        'f' -- file system (FILESYS)
        'a' -- address spaces (USER_PROGRAM)
        'n' -- network emulation (NETWORK)

9.  ASSERT : If condition is false,  print a message and dump core. Useful for documenting assumptions in the code.
    DEBUG  : Print debug message if flag is enabled.

10. 

11. 
// The following class defines a "list" -- a singly linked list of
// list elements, each of which points to a single item on the list.
//
// By using the "Sorted" functions, the list can be kept in sorted
// in increasing order by "key" in ListElement.

// The following class defines a "synchronized list" -- a list for which:
// these constraints hold:
//	1. Threads trying to remove an item from a list will
//	wait until the list has an element on it.
//	2. One thread at a time can access list data structures

12.

/code/test/matmult.c:main()
/code/test/halt.c:main()
/code/test/filetest.c:main()
/code/test/sort.c:main()
/code/test/shell.c:main()

/code/bin/coff2noff.c:main (int argc, char **argv)
/code/bin/coff2flat.c:main (int argc, char **argv)

/code/threads/main.cc:main(int argc, char **argv)


Entendemos que el archivos en el que esta la funcion main del ejecutable de Nachos del direcctorio userprog es:
/code/threads/main.cc:main(int argc, char **argv)

13.

Las lineas de comando que soporta Nachos son:

// Usage: nachos -d <debugflags> -rs <random seed #>
//		-s -x <nachos file> -c <consoleIn> <consoleOut>
//		-f -cp <unix file> <nachos file>
//		-p <nachos file> -r <nachos file> -l -D -t
//              -n <network reliability> -m <machine id>
//              -o <other machine id>
//              -z
//
//    -d causes certain debugging messages to be printed (cf. utility.h)
//    -rs causes Yield to occur at random (but repeatable) spots
//    -z prints the copyright message
//
//  USER_PROGRAM
//    -s causes user programs to be executed in single-step mode
//    -x runs a user program
//    -c tests the console
//u
//  FILESYS
//    -f causes the physical disk to be formatted
//    -cp copies a file from UNIX to Nachos
//    -p prints a Nachos file to stdout
//    -r removes a Nachos file from the file system
//    -l lists the contents of the Nachos directory
//    -D prints the contents of the entire file system 
//    -t tests the performance of the Nachos file system
//
//  NETWORK
//    -n sets the network reliability
//    -m sets this machine's host id (needed for the network)
//    -o runs a simple test of the Nachos network software
//
//  NOTE -- flags are ignored until the relevant assignment.
//  Some of the flags are interpreted here; some in system.cc.

El comando -rs produce que se llame al metodo Yield aliatoriamente dentro de la ejecucion. El metodo Yield lo que hace es detener la ejecucion del thread actual y selecciona uno nuevo de la lista de thread listos para ejecutar, si no hay ninguno otro listo, sigue ejecutando el thread actual.

14.
void
ThreadTest()
{
    DEBUG('t', "Entering SimpleTest");
    
    int i;
    Thread* newThread;

    for(i=0; i < 5; i++) {
        char *threadname = new char[128];
        stringstream ss;
        string aux("Hilo ");

        ss << i;
        string str = ss.str();
        aux += str;

        strcpy(threadname, aux.c_str());
        newThread = new Thread (threadname);
        newThread->Fork (SimpleThread, (void*)threadname);
    }
}

15.

\end{document}
