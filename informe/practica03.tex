\chapter{Resolución de la Práctica N\grad3: Programas de usuario y Multi-Programación}
\section{Desarrollo del API para copiar datos desde el núcleo al espacio de memoria del usuario y viceversa}
La interfaz \texttt{usertranslate.h} contiene las funciones que integran la API para traducir \texttt{Strings} y \texttt{Arrays}. La API contiene las siguientes funciones:
\begin{lstlisting}[style=C]
// Lee una string desde la memoria de usuario.
void readStrFromUsr(int usrAddr, char *outStr);
// Lee un array desde la memoria del usuario.
void readBuffFromUsr(int usrAddr, char *outBuff, int byteCount);
// Traduce una string hacia la memoria de usuario.
void writeStrToUsr(char *str, int usrAddr);
// Traduce un array hacia la memoria de usuario.
void writeBuffToUsr(char *str, int usrAddr, int byteCount);
\end{lstlisting}
Para implementarlas, se utilizaron los métodos \texttt{Machine::ReadMemory} y \texttt{Machine::WriteMemory}. A continuación, se muestran las implementación de la API:
\begin{lstlisting}[style=C]
void
readStrFromUsr(int usrAddr, char *outStr) {
    int value, count = 0;
    bool done;
   
    done = machine->ReadMem(usrAddr, 1, &value);
    ASSERT(done);
                
    while((char) value != '\0'){
        outStr[count] = (char) value;
        count++;
        done = machine->ReadMem(usrAddr + count, 1, &value);
        ASSERT(done);
    }
    outStr[count] = '\0';
}
\end{lstlisting}
\begin{lstlisting}[style=C]
void
readBuffFromUsr(int usrAddr, char *outBuff, int byteCount) {
    int value;
    bool done;
    for(int i=0; i < byteCount; i++) {
        done = machine->ReadMem(usrAddr+i, 1, &value);
        ASSERT(done);
        outBuff[i] = (char) value;
    }
}
\end{lstlisting}
\begin{lstlisting}[style=C]
void
writeStrToUsr(char *str, int usrAddr) {
    bool done;
    while(*str != '\0') {
        done = machine->WriteMem(usrAddr, 1, *(str));
        ASSERT(done);
        usrAddr++;
        str++;
    }
}
\end{lstlisting}
\begin{lstlisting}[style=C]
void
writeBuffToUsr(char *str, int usrAddr, int byteCount) {
    bool done;
    for(int i=0; i < byteCount; i++) {
        done = machine->WriteMem(usrAddr + i, 1, (int) str[i]);
        ASSERT(done);
    }
}
\end{lstlisting}
\section{Implementación de las llamadas de sistema y la administración de interrupciones}
Los programas de usuario invocan llamadas al sistema ejecutando la instrucción \texttt{syscall} de \textbf{\textit{MIPS}}, la cuál genera una trampa de hardware en el kernel de \textbf{\textit{NachOS}}. El simulador \textbf{\textit{NachOS/MIPS}} implementa trampas invocando al método \texttt{RaiseException}, pasándole argumentos que indican la causa exacta de la trampa. \texttt{RaiseException}, a su vez, llama a \texttt{ExceptionHandler} para que se ocupe del problema específico.\\
Por convención, los programas de usuario colocan el código que indica la llamada de sistema deseada en el registro \texttt{R2} antes de ejecutar la instrucción \texttt{syscall}. Mientras que los argumentos adicionales se encuentran en los registros \texttt{R4 a R7}. Se espera que los valores de retorno de la función (y de la llamada del sistema) estén en el registro \texttt{R2} al regresar.\\
Para poder implementarla las llamadas de sistema, se crearon las siguientes funciones y estructuras auxiliares:
\begin{itemize}
    \item \texttt{type}: en esta variable se guarda el tipo de llamada.
    \item \texttt{arguments}: aquí se guardan las direcciones de los argumentos.
    \item \texttt{result}: en esta variable se guarda el resultado de procesar la llamada. Luego, se escribirá dicho valor en el registro \texttt{R2}.
    \item \texttt{movingPC}: este método actualiza el PC para mantener el correcto funcionamiento del stack de registros. Se invoca al finalizar el procesamiento de una llamada a sistema.
        \begin{lstlisting}[style=C]
void
movingPC()
{
     int pc = machine->ReadRegister(PCReg);
     machine->WriteRegister(PrevPCReg, pc);
     pc = machine->ReadRegister(NextPCReg);
     machine->WriteRegister(PCReg, pc);
     pc += 4;
     machine->WriteRegister(NextPCReg, pc);
}
        \end{lstlisting}
\end{itemize}
El procesamiento de las llamadas a sistema quedó de la siguiente manera:
\begin{lstlisting}[style=C]
void
ExceptionHandler(ExceptionType which)
{
    int type = machine->ReadRegister(2);
    int arguments[4];
    arguments[0] = machine->ReadRegister(4);
    arguments[1] = machine->ReadRegister(5);
    arguments[2] = machine->ReadRegister(6);
    arguments[3] = machine->ReadRegister(7);
    int result;
    OpenFile* file;
    char name386[128];

    if (which == SyscallException) {
    	switch(type) {
            case SC_Halt:
                ...
                break;
            case SC_Create:
                break;
            case SC_Exit:
                break;
            case SC_Exec:
                break;
            case SC_Join:
                break;
            case SC_Open:
                break;
            case SC_Read:
                break;
            case SC_Write:
                break;
            case SC_Close:
                break;
            default: 
                printf("Unexpected syscall exception %d %d\n", which, type);
                ASSERT(false);
    	}
        machine->WriteRegister(2, result);
    	movingPC();
    } else {
        DEBUG('e', "Is not a SyscallException\n");
        printf("Unexpected user mode exception:\t which=%s  type=%d\n", exception, type);
        ASSERT(false);
    }
  }
\end{lstlisting}
En las próximas subsecciones se muestra cómo se desarrollaron las llamadas de sistema. Las mismas fueron implementadas en el orden propuesto por la cátedra.
\subsection*{Consola sincrónica}
Se crea la clase \texttt{SynchConsole}, la cual provee una abstracción de acceso sincronizado a la consola. Un requisito que cumple es que un hilo queriendo escribir no bloquea a un hilo queriendo leer.
\begin{lstlisting}[style=C]
class SynchConsole {
public:
	SynchConsole(const char *readFile, const char *writeFile);
	~SynchConsole();
	
    void WriteConsole(char c); // Write a char into console.
    char ReadConsole(); // Read a char from console.
    void RequestWrite();    // Provides sync access for reading
                            //the console.
    void RequestRead();     // Provides sync access for writing
                            //the console.
private: 
    Console *console;   // NacOS console.
    Semaphore *readAvail,   // For reader console handler.
              *writeDone;   // For writer console handler.
    Lock *writer,   // For sync writing access.
         *reader;   // For sync reading access.
};
\end{lstlisting}
Para desarrollarla, se tuvieron en cuenta las clases \texttt{progtest.cc} y \texttt{synchdisc.cc}.\\
Los semáforos \texttt{cread} y \texttt{cwrite}, permiten el acceso sincronizado a la lectura y escritura de la consola.\\
Cuando se levanta el hilo principal de \textbf{\textit{NachOS}}, se crea una consola de acceso sincronizado. Ademas, se reservan dos descriptores para representar la entrada y la salida estándar. Dichos descriptores son el \texttt{0} y el \texttt{1}. En \texttt{syscall.h} se definen dos constantes que modelan lo mencionado anteriormente:
\begin{lstlisting}
#define CONSOLE_INPUT	0
#define CONSOLE_OUTPUT	1
\end{lstlisting}
\newpage
\subsection*{SC\_Create}
Dado un nombre de archivo, su función es crearlo.
\begin{lstlisting}[style=C]
void Create(char *name);
\end{lstlisting}
Recibe un parámetro, el cual es la dirección de memoria donde se aloja un string para indicar el nombre del archivo. Lo traduce mediante \texttt{readStrFromUsr} e invoca a \texttt{FileSystem::Create}.\\
A través de \texttt{R2}, devuelve \texttt{0} en caso de crear el archivo; sino devuelve \texttt{-1}.
\subsection*{Read}
Dado un descriptor de un archivo \texttt{id}, lee \texttt{size} bytes y los guarda en el array \texttt{buffer}. Devuelve la cantidad de bytes leídos.
\begin{lstlisting}[style=C]
int Read(char *buffer, int size, OpenFileId id);
\end{lstlisting}
Si el descriptor es \texttt{CONSOLE\_INPUT}, lee de la consola \texttt{size} caracteres o hasta que aparezca el caracter de salto de linea('\texttt{\textbackslash{n}}'). Para leerlos, se invoca a \texttt{synchConsole->ReadConsole}.\\
En caso de que el descriptor corresponde a un archivo previamente abierto por el usuario, se leen de dicho archivo \texttt{size} caracteres. Para leer desde un archivo, se invoca a \texttt{OpenFile::Read}.\\
En cambio, si el descriptor \texttt{id} no corresponde a ningún archivo abierto por el usuario, se imprime un mensaje de error y devuelve \texttt{-1}.\\
Una vez que es leído exitosamente, se invoca a la función \texttt{writeBuffToUsr} para traducir el resultado de la lectura a una dirección en el espacio de direcciones indicado por el usuario en el primer argumento(\texttt{R4}).
\subsection*{Write}
Dado un array \texttt{buffer}, escribe \texttt{size} bytes en el archivo cuyo descriptor es \texttt{id}. Devuelve la cantidad de bytes leídos.
\begin{lstlisting}[style=C]
void Write(char *buffer, int size, OpenFileId id);
\end{lstlisting}
Es similar a la llamada \texttt{SC\_Read}. Se puede escribir en la consola sincrónica o en un archivo.\\
Al principio, se lee una dirección del espacio de memoria del usuario, la cual contiene la información a escribir. Para ello, se traduce el primer argumento invocando a \texttt{readBuffFromUsr} y guardándolo en la variable \texttt{buffer}.\\
Si el descriptor es \texttt{CONSOLE\_OUTPUT}, escribe en la consola \texttt{size} caracteres. Para ello, invoca a \texttt{synchConsole->WriteConsole}.\\
En caso de que el descriptor corresponde a un archivo previamente abierto por el usuario, se escribe en dicho archivo \texttt{size} caracteres, almacenados en \texttt{buffer}. Para escribir en un archivo, se invoca a \texttt{OpenFile::Write}.\\
En cambio, si el descriptor \texttt{id} no corresponde a ningún archivo abierto por el usuario, se imprime un mensaje de error y devuelve \texttt{-1}.
\subsection*{Open}
Dado un nombre de un archivo, lo abre y retorna su descriptor.
\begin{lstlisting}[style=C]
OpenFileId Open(char *name);
\end{lstlisting}
Al principio, se traduce el primer argumento invocando a \texttt{readStrFromUsr}. A continuación, se invoca a \texttt{OpenFile::Open} para abrir un archivo. Si la apertura es exitosa, se le indica al \texttt{currentThread} que abrió dicho archivo, mediante la invocación a \texttt{Thread::AddFile}. Luego, se retorna el descriptor recibido. Si falla la apertura, se retorna \texttt{-1}.\\
Por otro lado, cabe aclarar que una restricción que tiene un programa al manejar archivos, es la cantidad máxima de archivos que puede tener abiertos al mismo tiempo. Para modelar ello, se definió la siguiente constante en \texttt{threads.h}:
\begin{lstlisting}[style=C]
#define MAX_FILES_OPENED 5
\end{lstlisting}
Otra restricción que tiene un programa al manejar archivos es que sólo puede leer o escribir los archivos abiertos por él. Para modelar esto, se creo en \texttt{threads.h} una estructura de control para llevar un registro de los archivos abiertos por cada programa usuario.
\begin{lstlisting}[style=C]
class Thread {
...
#ifdef USER_PROGRAM
private:
...
 OpenFile* filesDescriptors[MAX_FILES_OPENED]; // Structure for maintain all files opened for this user program.

public:
...
    OpenFile* GetFile(OpenFileId descriptor); // Given a descriptor, returns the corresponding file.
    OpenFileId AddFile(OpenFile* file); // Loads a file, assing an descriptor and return its.
    void RemoveFile(OpenFileId descriptor); // Given a descriptor, unload its corresponding file.
#endif
\end{lstlisting}
Dicha estructura consta de un arreglo de archivos llamado \texttt{filesDescriptors}, de tamaño \texttt{MAX\_FILES\_OPENDED} y tres métodos para interactuar con dicha estructura:
\begin{description}
    \item[Thread::AddFile] Dado un archivo, lo guarda en el arreglo y devuelve un \texttt{OpenFileId}. Guardamos cada archivo en el indice del arreglo, correspondiente a su descriptor. Para buscar un descriptor nuevo, basta con localizar un lugar vacío en el arreglo y devolver su posición. Se reservan el \texttt{0} y el \texttt{1} para la entrada y la salida estándar. Si no hay espacio libre, retorna \texttt{-1}.
    \item[Thread::GetFile] Dado un descriptor de un archivo, retorna el archivo correspondiente. Para ello, busca en el arreglo. Si no lo encuentra, devuelve \texttt{NULL}.
    \item[Thread::RemoveFile] Dado un descriptor de un archivo, limpia la posición dada.
\end{description}
\subsection*{Close}
Dado un descriptor de un archivo, lo cierra.
\begin{lstlisting}[style=C]
void Close(OpenFileId id);
\end{lstlisting}
Recibe como argumento el descriptor del archivo a cerrar. Con él, obtiene el archivo invocando \texttt{Thread::GetFile}. En caso de obtener un archivo, significa que fue abierto por el usuario. Entonces lo elimina y lo quita. Para ello, invoca a \texttt{OpenFile::~OpenFile} y \texttt{Thread::RemoveFile}, respectivamente. Luego retorna \texttt{0}.\\
En caso contrario, imprime el error en pantalla y retorna \texttt{-1}.\\
\textbf{Test para manejo de archivos}\\
\framebox{
    \begin{minipage}{\textwidth}
        \begin{description}
            \item \texttt{testConsole.c}: Lee de la consola caracteres hasta llegar a 255 o hasta presionar \texttt{ENTER}. Luego, imprime los caracteres leídos en la consola.
            \item \texttt{writetest.c}: Crea un archivo llamado \texttt{writetest.txt}. Lo abre, escribe un mensaje dentro y lo cierra.
            \item \texttt{readtest.c}: Primero abre el archivo \texttt{readtest.txt}. Lee los primeros 100 caracteres y los imprime en la consola. Luego cierra el archivo.
        \end{description}
    \end{minipage}
}
\subsection*{Exit}
Finaliza el programa según el estado dado.
\begin{lstlisting}[style=C]
void Exit(int status);
\end{lstlisting}
Si el argumento es \texttt{0}, termina correctamente. Sino imprime un cartel advirtiendo que el programa tuvo problemas.
\subsection*{Join}
Dado un id de un proceso, espera a que termine y luego continua su ejecución.
\begin{lstlisting}[style=C]
int Join(SpaceId id);
\end{lstlisting}
Para poder esperar a un proceso, se tiene que llevar cuenta de los procesos que se están ejecutando. Por ello, creamos la clase \texttt{ProcessTable} que abstrae la relación mencionada anteriormente.
\begin{lstlisting}[style=C]
class ProcessTable{
  public:
    ProcessTable();
    ~ProcessTable();	// De-allocate an process table

    void addProcess(SpaceId pid, Thread* executor); // Add process at the table.
    int getFreshSlot(); // Returns an empty slot of the table.
    Thread* getProcess(SpaceId pid); // Given a pid, returns the appropiate process.
    void freeSlot(SpaceId pid);		// Delete process from the table.

  private:
    Thread** table;
};
\end{lstlisting}
Se agregó una variable llamada \texttt{processTable} en \texttt{system.h}. La tabla se crear cuando se inicializa el sistema. Ademas, se agregó una cota máxima de procesos que se ejecutan al mismo tiempo. Esto define la cantidad de entradas que tiene la tabla y se encuentra en \texttt{processtable.h}.
\begin{lstlisting}[style=C]
#define MAX_EXEC_THREADS 20
\end{lstlisting}
Todos los procesos que se ejecutan en modo usuario, tienen que estar registrados en dicha tabla. En particular, el proceso 'Main' también tiene que registrarse. Por ello, se agrega a la tabla momentos después de crearse, en el método \texttt{Inicialize}.\\
Volviendo a la implementación de la llamada a sistema \texttt{Join}, utilizando el argumento se invoca a \texttt{ProcessTable::getProcess} para obtener el thread que ejecutó dicho id.\\
Si es distinto de \texttt{NULL}, le hace Join invocando a \texttt{Thread::Join}. Caso contrario, imprime un mensaje de error y retorna \texttt{-1}.
\subsection*{Exec sin argumentos} \label{exec_without_args}
Dado un archivo ejecutable, lo ejecuta. Para ejecutar un programa sin argumentos, se lo invoca con \texttt{0} y \texttt{'\textbackslash 0'} en el segundo y el tercer parámetro, respectivamente.
\begin{lstlisting}[style=C]
SpaceId Exec(char *name, int argc, char** argv);
\end{lstlisting}
Inicialmente, traducimos el nombre del archivo mediante \texttt{readStrFromUsr} e intentamos abrirlo invocando \texttt{OpenFile::Open}. Si falla, imprimimos un mensaje de error y retornamos el valor \texttt{-1}.\\
En caso de poder abrir el archivo, obtenemos un id para el nuevo proceso. Para ello, invocamos a \texttt{ProcessTable::GetFreshSlot}. Luego, se crea un thread para ejecutar el archivo y se asigna un espacio de direcciones y se carga el archivo ejecutable en memoria. Todo ello esta modularizado en la función \texttt{makeProcess}.
\begin{lstlisting}[style=C]
void
makeProcess(int pid, OpenFile* executable, char* filename, int argc, char** argv)
{
    DEBUG ('e',"C: currenThread:%s\t pid %d\t filename=%s\n", currentThread->getName(),
           pid, filename);
    
    Thread *execThread = new Thread(filename, 0);    //Creation of thread executor.
    execThread->setThreadId(pid);
    processTable->addProcess(pid, execThread);

    // Creation of space address for process.
    AddrSpace *execSpace = new AddrSpace(executable, argc, argv);
    execThread->space = execSpace;
    delete executable;

    amountThread++;
    execThread->Fork(doExecution, (void*) filename, 1);   //Create process.

    DEBUG('e', "After finish makeProcess, pid=%d\t name=$s\n", execThread->getThreadId(),
          execThread->getName());
}
\end{lstlisting}
Cuando el \textbf{Scheduler} ejecute el thread \texttt{execThread}, va inicializar su espacio de memoria y va a llamar a la maquina MIPS de \texttt{NachOS}, como vemos en el cuerpo de la función \texttt{doExecution}.
\begin{lstlisting}[style=C]
doExecution(void* arg)
{
    DEBUG ('e',"doExecution: currentThread name=%s\t id=%d\n", currentThread->getName(),
           currentThread->getThreadId());

	currentThread->space->InitRegisters();  //Inicialization for MIPS registers.
	currentThread->space->RestoreState();   //Load page table register.

 	machine->Run();	

    ASSERT(false);  //Machine->Run never returns.
}
\end{lstlisting}
\section{Multiprogramación con rebanadas de tiempo (time-slicing)}
\subsection*{Múltiples programas de usuario}
Para soportar la multiprogramación, proponemos utilizar una estructura llamada \textbf{Bitmap} (definida en \texttt{bitmap.h}) para administrar los marcos de la memoria física; de tal forma que podamos saber cuales están asignados y cuales están libres. Dicho mapa de bits va a ser una variable global llamada \texttt{memoryMap} y se define en \texttt{system.cc}, pasándole como argumento el numero de paginas físicas (\texttt{NumPhysPages}).\\
A continuación, modificamos la clase \texttt{AddrSpace} para que soporte la multiprogramación:
\begin{itemize}
    \item \textbf{Chequeo de espacio}: Se modifico el calculo de cuanto espacio necesita un archivo ejecutable en memoria para que soporte el paginado.
    \begin{lstlisting}[style=C]
size = noffH.code.size + noffH.initData.size 
       + noffH.uninitData.size + UserStackSize;
numPages = divRoundUp(size, PageSize);
size = numPages * PageSize;
    \end{lstlisting}
    \item \textbf{Asegurarse de que hay espacio disponible}: Luego de calcular la cantidad de paginas que necesitamos, tenemos que chequear que tengamos libre la cantidad suficiente.
    \item \textbf{Guardar el marco en cada pagina}: Cuando se arma cada pagina, guardamos en \texttt{physicalAddr} el marco correspondiente a la memoria física. Dicho marco, es uno libre que se obtiene mediante el mapa de bits.
    \begin{lstlisting}[style=C]
int firstFreePhySpace = -1;
for (i = 0; i < numPages; i++) {
    firstFreePhySpace = memoryMap->Find();
    ASSERT(firstFreePhySpace != -1);	//Always found space in physical memory.
    pageTable[i].physicalPage = firstFreePhySpace;
    ...
}
    \end{lstlisting}
    \item \textbf{Inicializa la pagina}: Se inicializa cada pagina a traves del metodo \texttt{bzero}.
    \begin{lstlisting}
for (i = 0; i < numPages; i++) {
    ...
    bzero(&(machine->mainMemory[firstFreePhySpace*PageSize]), PageSize);
}
    \end{lstlisting}
    \item \textbf{Carga de segmentos}: Finalmente, se cargan los segmentos del archivo ejecutable en memoria. Dicho proceso, se modularizo en el método \texttt{LoadSegment}, cuya función es cargar un segmento en memoria.
\end{itemize}
\subsection*{Cambio de contexto con rebanadas de tiempo}
Para forzar los cambios de contexto, se modificó la creación del \texttt{timer}, al momento de iniciar \textit{\textbf{NachOS}}.
\begin{lstlisting}[style=C]
#ifdef USER_PROGRAM
    randomYield = false;
#endif
    timer = new Timer(TimerInterruptHandler, 0, randomYield);
\end{lstlisting}
Al invocar el constructor con  el tercer parámetro en \texttt{false}, se crea un temporizador cuyo timeout ocurrirá cada \texttt{TimerTicks} unidades. La definición de \texttt{TimerTicks} se encuentra en \texttt{stats.h}.
\section{Exec con argumentos}
A lo desarrollado en \nameref{exec_without_args}, se agrego el procesamiento del segundo y tercer argumento. La cantidad de argumentos nos viene en el registro \texttt{R5} y el string de argumentos sin traducir, vienen por el registro \texttt{R6}. Para traducir el vector de argumentos, invocamos a método \texttt{readSpecialStringFromUser}, declarado en \texttt{usertranlate.h}, para traducir cada argumento y guardarlos en el array \texttt{argv}. Por convención, el string con los argumentos estará separado por espacios.
\begin{lstlisting}[style=C]
// Dado un string con argumentos separados por 'divide',
//traduce un argumento y retorna la direccion del proximo
//argumento a traducir.
int readSpecialStringFromUser(int usrAddr,char *outStr,char divide);
\end{lstlisting}
Al finalizar la traducción, se invoca a \texttt{makeProcess} y se continua como mencionamos en \nameref{exec_without_args}. Ademas, se modifico la función \texttt{doExecution} para que inicialice los argumentos antes de ejecutar el archivo.\\
Por otro lado, se modifico la clase \texttt{AddSpace}, ya que necesitamos copiar los argumentos en el stack de usuario. En la llamada al constructor, se guardan \texttt{argc} y \texttt{argv}. Tambien se creo el método \texttt{AddrSpace::InitArguments}, el cual se encarga de almacenar correctamente los argumentos en el stack. Empieza guardando en la parte superior de la pila el valor de los argumentos, seguidos los punteros a esas direcciones. Finalmente, mueve el stackpointer cuatro posiciones mas abajo del ultimo argumento ingresado en la pila. 
\section{Interpretes de comandos y programas de usuario}
Se implemento un interprete de comandos, el cual lee un comando de la consola y lo ejecuta. Si el comando comienza con el carácter \texttt{\&}, debe ser ejecutado en background. En \texttt{test/shell.c} se encuentra la implementación del interprete.\\
Ademas, se implementaron los siguientes programas utilitarios:
\begin{description}
    \item \textbf{cat} Su función es leer un archivo e imprimirlo en la consola.
    \item \textbf{cp} Su función es copiar el contenido de un archivo a otro. Para ello, necesita el nombre del archivo a copiar y el nombre del archivo destino.
    \item \textbf{clear} Su función es limpiar la consola. Para ello, imprime 250 saltos de linea.
\end{description}
Las implementaciones de todos ellos se encuentran la carpeta \texttt{'test/'}.