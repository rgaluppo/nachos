\chapter{Resolución de la Práctica N\grad3: Programas de usuario y Multi-Programación}
\section{Desarrollo del API para copiar datos desde el núcleo al espacio de memoria del usuario y viceversa}
La interfaz \texttt{usertranslate.h} contiene las funciones que integran la API para traducir \texttt{Strings} y \texttt{Arrays}. La API contiene las siguientes funciones:
\begin{lstlisting}[style=C]
    // Lee una string desde la memoria de usuario.
    void readStrFromUsr(int usrAddr, char *outStr);
    // Lee un array desde la memoria del usuario.
    void readBuffFromUsr(int usrAddr, char *outBuff, int byteCount);
    // Traduce una string hacia la memoria de usuario.
    void writeStrToUsr(char *str, int usrAddr);
    // Traduce un array hacia la memoria de usuario.
    void writeBuffToUsr(char *str, int usrAddr, int byteCount);
\end{lstlisting}
Para implementarlas, se utilizaron los métodos \texttt{Machine::ReadMemory} y \texttt{Machine::WriteMemory}. A continuación, se muestran las implementación de la API:
\begin{lstlisting}[style=C]
//-----------------------------------------------------------
// readStrFromUsr
//  Read a string from user memory space.
//
// usrAddr memory address from user space.
// outStr string where is the result of translation.
//-----------------------------------------------------------
void
readStrFromUsr(int usrAddr, char *outStr) {
    int value, count = 0;
    bool done;
   
    done = machine->ReadMem(usrAddr, 1, &value);
    ASSERT(done);
                
    while((char) value != '\0'){
        outStr[count] = (char) value;
        count++;
        done = machine->ReadMem(usrAddr + count, 1, &value);
        ASSERT(done);
    }
    outStr[count] = '\0';
}
\end{lstlisting}
\begin{lstlisting}[style=C]
//-----------------------------------------------------------
// readBuffFromUsr
//  Read a buffer from user memory space.
//
// usrAddr memory address from user space.
// outBuff array where is the result of translation.
// byteCount amount of bytes reads.
//-----------------------------------------------------------
void
readBuffFromUsr(int usrAddr, char *outBuff, int byteCount) {
    int value;
    bool done;
    for(int i=0; i < byteCount; i++) {
        done = machine->ReadMem(usrAddr+i, 1, &value);
        ASSERT(done);
        outBuff[i] = (char) value;
    }
}
\end{lstlisting}
\begin{lstlisting}[style=C]
//-----------------------------------------------------------
// writeStrToUsr
//  Translate a string to user memory space.
//
// str string where that will be translated.
// usrAddr memory address to user space where will be located
//         the beginig of tralated string.
//-----------------------------------------------------------
void
writeStrToUsr(char *str, int usrAddr) {
    bool done;
    while(*str != '\0') {
        done = machine->WriteMem(usrAddr, 1, *(str));
        ASSERT(done);
        usrAddr++;
        str++;
    }
}
\end{lstlisting}
\begin{lstlisting}[style=C]
//-----------------------------------------------------------
// writeBuffToUsr
//  Translate an array to user memory space.
//
// str string where that will be translated.
// usrAddr memory address to user space where will be located
//         the beginig of tralated array.
// byteCount amount of bytes reads.
//-----------------------------------------------------------
void
writeBuffToUsr(char *str, int usrAddr, int byteCount) {
    bool done;
    for(int i=0; i < byteCount; i++) {
        done = machine->WriteMem(usrAddr + i, 1, (int) str[i]);
        ASSERT(done);
    }
}
\end{lstlisting}
\section{Implementación de las llamadas de sistema y la administración de interrupciones}
Los programas de usuario invocan llamadas al sistema ejecutando la instrucción \texttt{syscall} de \textbf{\textit{MIPS}}, la cuál genera una trampa de hardware en el kernel de \textbf{\textit{NachOS}}. El simulador \textbf{\textit{NachOS/MIPS}} implementa trampas invocando al método \texttt{RaiseException}, pasándole argumentos que indican la causa exacta de la trampa. \texttt{RaiseException}, a su vez, llama a \texttt{ExceptionHandler} para que se ocupe del problema específico.\\
Por convención, los programas de usuario colocan el código que indica la llamada de sistema deseada en el registro \texttt{R2} antes de ejecutar la instrucción \texttt{syscall}. Mientras que los argumentos adicionales se encuentran en los registros \texttt{R4 a R7}. Se espera que los valores de retorno de la función (y de la llamada del sistema) estén en el registro \texttt{R2} al regresar.\\
Para poder implementarla las llamadas de sistema, se crearon las siguientes funciones y estructuras auxiliares:
\begin{itemize}
    \item \texttt{type}: en esta variable se guarda el tipo de llamada.
    \item \texttt{arguments}: aquí se guardan las direcciones de los argumentos.
    \item \texttt{result}: en esta variable se guarda el resultado de procesar la llamada. Luego, se escribirá dicho valor en el registro \texttt{R2}.
    \item \texttt{movingPC}: este método actualiza el PC para mantener el correcto funcionamiento del stack de registros. Se invoca al finalizar el procesamiento de una llamada a sistema.
        \begin{lstlisting}[style=C]
void
movingPC()
{
     int pc = machine->ReadRegister(PCReg);
     machine->WriteRegister(PrevPCReg, pc);
     pc = machine->ReadRegister(NextPCReg);
     machine->WriteRegister(PCReg, pc);
     pc += 4;
     machine->WriteRegister(NextPCReg, pc);
}
        \end{lstlisting}
\end{itemize}
El procesamiento de las llamadas a sistema quedó de la siguiente manera:
\begin{lstlisting}[style=C]
void
ExceptionHandler(ExceptionType which)
{
    int type = machine->ReadRegister(2);
    int arguments[4];
    arguments[0] = machine->ReadRegister(4);
    arguments[1] = machine->ReadRegister(5);
    arguments[2] = machine->ReadRegister(6);
    arguments[3] = machine->ReadRegister(7);
    int result;
    OpenFile* file;
    char name386[128];

    if (which == SyscallException) {
    	switch(type) {
            case SC_Halt:
                ...
                break;
            case SC_Create:
                break;
            case SC_Exit:
                break;
            case SC_Exec:
                break;
            case SC_Join:
                break;
            case SC_Open:
                break;
            case SC_Read:
                break;
            case SC_Write:
                break;
            case SC_Close:
                break;
            default: 
                printf("Unexpected syscall exception %d %d\n", which, type);
                ASSERT(false);
    	}
        machine->WriteRegister(2, result);
    	movingPC();
    } else {
        DEBUG('e', "Is not a SyscallException\n");
        printf("Unexpected user mode exception:\t which=%s  type=%d\n", exception, type);
        ASSERT(false);
    }
  }
\end{lstlisting}

En las próximas subsecciones se muestra cómo se desarrollaron las llamadas de sistema. Las mismas fueron implementadas en el orden propuesto por la cátedra.\\
\subsection*{SC\_Create}
\begin{lstlisting}[style=C]
void Create(char *name);
\end{lstlisting}
Su función es crear un archivo. Recibe un parámetro, el cual es la dirección de memoria donde se aloja un string para indicar el nombre del archivo. Lo traduce mediante \texttt{readStrFromUsr} e invoca a \texttt{FileSystem::Create}.\\
A través de \texttt{R2}, devuelve \texttt{0} en caso de crear el archivo; sino devuelve \texttt{-1}.
\subsection*{Consola sincrónica}
Se crea la clase \texttt{SynchConsole}, la cual provee una abstracción de acceso sincronizado a la consola. Un requisito que cumple es que un hilo queriendo escribir no bloquea a un hilo queriendo leer.
\begin{lstlisting}[style=C]
class SynchConsole {
public:
	SynchConsole(const char *readFile, const char *writeFile);
	~SynchConsole();
	
    void WriteConsole(char c); // Write a char into console.
    char ReadConsole(); // Read a char from console.
    void RequestWrite();    // Provides sync access for reading
                            //the console.
    void RequestRead();     // Provides sync access for writing
                            //the console.
private: 
    Console *console;   // NacOS console.
    Semaphore *readAvail,   // For reader console handler.
              *writeDone;   // For writer console handler.
    Lock *writer,   // For sync writing access.
         *reader;   // For sync reading access.
};
\end{lstlisting}
Para desarrollarla, se tuvieron en cuenta las clases \texttt{progtest.cc} y \texttt{synchdisc.cc}.\\
Los semáforos \texttt{cread} y \texttt{cwrite}, permiten el acceso sincronizado a la lectura y escritura de la consola.\\
Cuando se levanta el hilo principal de \textbf{\textit{NachOS}}, se crea una consola de acceso sincronizado. Ademas, se reservan dos descriptores para representar la entrada y la salida estándar. Dichos descriptores son el \texttt{0} y el \texttt{1}. En \texttt{syscall.h} se definen dos constantes que modelan lo mencionado anteriormente:
\begin{lstlisting}
#define CONSOLE_INPUT	0
#define CONSOLE_OUTPUT	1
\end{lstlisting}
\subsection*{Read}
Dado un descriptor de un archivo \texttt{id}, lee \texttt{size} bytes y los guarda en el array \texttt{buffer}. Devuelve la cantidad de bytes leídos.
\begin{lstlisting}[style=C]
int Read(char *buffer, int size, OpenFileId id);
\end{lstlisting}
Si el descriptor es \texttt{CONSOLE\_INPUT}, lee de la consola \texttt{size} caracteres o hasta que aparezca el caracter de salto de linea('\texttt{\textbackslash{n}}'). Para leerlos, se invoca a \texttt{synchConsole->ReadConsole}.\\
En caso de que el descriptor corresponde a un archivo previamente abierto por el usuario, se leen de dicho archivo \texttt{size} caracteres. Para leer desde un archivo, se invoca a \texttt{OpenFile::Read}.\\
En cambio, si el descriptor \texttt{id} no corresponde a ningún archivo abierto por el usuario, se imprime un mensaje de error y devuelve \texttt{-1}.\\
Una vez que es leído exitosamente, se invoca a la función \texttt{writeBuffToUsr} para traducir el resultado de la lectura a una dirección en el espacio de direcciones indicado por el usuario en el primer argumento(\texttt{R4}).
\subsection*{Write}
Dado un array \texttt{buffer}, escribe \texttt{size} bytes en el archivo cuyo descriptor es \texttt{id}. Devuelve la cantidad de bytes leídos.\\
\begin{lstlisting}[style=C]
void Write(char *buffer, int size, OpenFileId id);
\end{lstlisting}
Es similar a la llamada \texttt{SC\_Read}. Se puede escribir en la consola sincrónica o en un archivo.\\
Al principio, se lee una dirección del espacio de memoria del usuario, la cual contiene la información a escribir. Para ello, se traduce el primer argumento invocando a \texttt{readBuffFromUsr} y guardándolo en la variable \texttt{buffer}.\\
Si el descriptor es \texttt{CONSOLE\_OUTPUT}, escribe en la consola \texttt{size} caracteres. Para ello, invoca a \texttt{synchConsole->WriteConsole}.\\
En caso de que el descriptor corresponde a un archivo previamente abierto por el usuario, se escribe en dicho archivo \texttt{size} caracteres, almacenados en \texttt{buffer}. Para escribir en un archivo, se invoca a \texttt{OpenFile::Write}.\\
En cambio, si el descriptor \texttt{id} no corresponde a ningún archivo abierto por el usuario, se imprime un mensaje de error y devuelve \texttt{-1}.\\
\subsection*{Open}
Dado un nombre de un archivo, lo abre y retorna su descriptor.\\
\begin{lstlisting}[style=C]
OpenFileId Open(char *name);
\end{lstlisting}
Al principio, se traduce el primer argumento invocando a \texttt{readStrFromUsr}. A continuación, se invoca a \texttt{OpenFile::Open} para abrir un archivo. Si la apertura es exitosa, se le indica al \texttt{currentThread} que abrió dicho archivo, mediante la invocación a \texttt{Thread::AddFile}. Luego, se retorna el descriptor recibido. Si falla la apertura, se retorna \texttt{-1}.\\
Por otro lado, cabe aclarar que una restricción que tiene un programa al manejar archivos, es la cantidad máxima de archivos que puede tener abiertos al mismo tiempo. Para modelar ello, se definió la siguiente constante en \texttt{threads.h}:
\begin{lstlisting}[style=C]
#define MAX_FILES_OPENED 5
\end{lstlisting}
Otra restricción que tiene un programa al manejar archivos es que sólo puede leer o escribir los archivos abiertos por él. Para modelar esto, se creo en \texttt{threads.h} una estructura de control para llevar un registro de los archivos abiertos por cada programa usuario.
\begin{lstlisting}[style=C]
class Thread {
...
#ifdef USER_PROGRAM
private:
...
 OpenFile* filesDescriptors[MAX_FILES_OPENED]; // Structure for maintain all files opened for this user program.

public:
...
    OpenFile* GetFile(OpenFileId descriptor); // Given a descriptor, returns the corresponding file.
    OpenFileId AddFile(OpenFile* file); // Loads a file, assing an descriptor and return its.
    void RemoveFile(OpenFileId descriptor); // Given a descriptor, unload its corresponding file.
#endif
\end{lstlisting}
Dicha estructura consta de un arreglo de archivos llamado \texttt{filesDescriptors}, de tamaño \texttt{MAX\_FILES\_OPENDED} y tres métodos para interactuar con dicha estructura:
\begin{description}
    \item[Thread::AddFile] Dado un archivo, lo guarda en el arreglo y devuelve un \texttt{OpenFileId}. Guardamos cada archivo en el indice del arreglo, correspondiente a su descriptor. Para buscar un descriptor nuevo, basta con localizar un lugar vacío en el arreglo y devolver su posición. Se reservan el \texttt{0} y el \texttt{1} para la entrada y la salida estándar. Si no hay espacio libre, retorna \texttt{-1}.
    \item[Thread::GetFile] Dado un descriptor de un archivo, retorna el archivo correspondiente. Para ello, busca en el arreglo. Si no lo encuentra, devuelve \texttt{NULL}.
    \item[Thread::RemoveFile] Dado un descriptor de un archivo, limpia la posición dada.
\end{description}
\subsection*{Close}
Dado un descriptor de un archivo, lo cierra.\\
\begin{lstlisting}[style=C]
void Close(OpenFileId id);
\end{lstlisting}
Recibe como argumento el descriptor del archivo a cerrar. Con él, obtiene el archivo invocando \texttt{Thread::GetFile}. En caso de obtener un archivo, significa que fue abierto por el usuario. Entonces lo elimina y lo quita. Para ello, invoca a \texttt{OpenFile::~OpenFile} y \texttt{Thread::RemoveFile}, respectivamente. Luego retorna \texttt{0}.\\
En caso contrario, imprime el error en pantalla y retorna \texttt{-1}.\\
\framebox{
    \begin{minipage}{\textwidth}
        \begin{description}
            \item \texttt{testConsole.c}: Lee de la consola caracteres hasta llegar a 255 o hasta presionar \texttt{ENTER}. Luego, imprime los caracteres leídos en la consola.
            \item \texttt{writetest.c}: Crea un archivo llamado \texttt{writetest.txt}. Lo abre, escribe un mensaje dentro y lo cierra.
            \item \texttt{readtest.c}: Primero abre el archivo \texttt{readtest.txt}. Lee los primeros 100 caracteres y los imprime en la consola. Luego cierra el archivo.
        \end{description}
    \end{minipage}
}
\subsection*{Exit}
\begin{lstlisting}[style=C]
void Exit(int status);
\end{lstlisting}
\subsection*{Join}
\begin{lstlisting}[style=C]
int Join(SpaceId id);
\end{lstlisting}
\subsection*{Exec}
\begin{lstlisting}[style=C]
SpaceId Exec(char *name, int argc, char** argv);
\end{lstlisting}
\subsection*{Test}
Esta sección menciona cuales test se hicieron para probar las llamadas a sistema. Todos ellos se encuentran en la carpeta \texttt{/test/}.
\begin{itemize}
    \item \texttt{testConsole.c}: Sirve para probar escribir y leer en la consola sincrónica.
\end{itemize}
\section{}
\section{}
\section{}