\chapter{Resolución de la Práctica N\grad3: Programas de usuario y Multi-Programación}
\section{Desarrollo del API para copiar datos desde el núcleo al espacio de memoria del usuario y viceversa}
La interfaz \texttt{usertranslate.h} contiene las funciones que integran la API para traducir \texttt{Strings} y \texttt{Arrays}. La API contiene las siguientes funciones:
\begin{lstlisting}[style=C]
    // Lee una string desde la memoria de usuario.
    void readStrFromUsr(int usrAddr, char *outStr);
    // Lee un array desde la memoria del usuario.
    void readBuffFromUsr(int usrAddr, char *outBuff, int byteCount);
    // Traduce una string hacia la memoria de usuario.
    void writeStrToUsr(char *str, int usrAddr);
    // Traduce un array hacia la memoria de usuario.
    void writeBuffToUsr(char *str, int usrAddr, int byteCount);
\end{lstlisting}
Para implementarlas, se utilizaron los métodos \texttt{Machine::ReadMemory} y \texttt{Machine::WriteMemory}. A continuación, se muestran las implementación de la API:
\begin{lstlisting}[style=C]
//-----------------------------------------------------------
// readStrFromUsr
//  Read a string from user memory space.
//
// usrAddr memory address from user space.
// outStr string where is the result of translation.
//-----------------------------------------------------------
void
readStrFromUsr(int usrAddr, char *outStr) {
    int value, count = 0;
    bool done;
   
    done = machine->ReadMem(usrAddr, 1, &value);
    ASSERT(done);
                
    while((char) value != '\0'){
        outStr[count] = (char) value;
        count++;
        done = machine->ReadMem(usrAddr + count, 1, &value);
        ASSERT(done);
    }
    outStr[count] = '\0';
}
\end{lstlisting}
\begin{lstlisting}[style=C]
//-----------------------------------------------------------
// readBuffFromUsr
//  Read a buffer from user memory space.
//
// usrAddr memory address from user space.
// outBuff array where is the result of translation.
// byteCount amount of bytes reads.
//-----------------------------------------------------------
void
readBuffFromUsr(int usrAddr, char *outBuff, int byteCount) {
    int value;
    bool done;
    for(int i=0; i < byteCount; i++) {
        done = machine->ReadMem(usrAddr+i, 1, &value);
        ASSERT(done);
        outBuff[i] = (char) value;
    }
}
\end{lstlisting}
\begin{lstlisting}[style=C]
//-----------------------------------------------------------
// writeStrToUsr
//  Translate a string to user memory space.
//
// str string where that will be translated.
// usrAddr memory address to user space where will be located
//         the beginig of tralated string.
//-----------------------------------------------------------
void
writeStrToUsr(char *str, int usrAddr) {
    bool done;
    while(*str != '\0') {
        done = machine->WriteMem(usrAddr, 1, *(str));
        ASSERT(done);
        usrAddr++;
        str++;
    }
}
\end{lstlisting}
\begin{lstlisting}[style=C]
//-----------------------------------------------------------
// writeBuffToUsr
//  Translate an array to user memory space.
//
// str string where that will be translated.
// usrAddr memory address to user space where will be located
//         the beginig of tralated array.
// byteCount amount of bytes reads.
//-----------------------------------------------------------
void
writeBuffToUsr(char *str, int usrAddr, int byteCount) {
    bool done;
    for(int i=0; i < byteCount; i++) {
        done = machine->WriteMem(usrAddr + i, 1, (int) str[i]);
        ASSERT(done);
    }
}
\end{lstlisting}
\section{Implementación de las llamadas de sistema y la administración de interrupciones}

\section{}
\section{}
\section{}