\chapter{Resolución de la Práctica N\grad2}
\section{Implementación de locks y variables de condición}
\subsection*{Intro}
Un cerrojo provee una exclusión mutua para utilizar ciertos recursos. Puede tener dos estados: libre y ocupado. Además, sólo se permiten dos operaciones:
\begin{itemize}
	\item \textbf{Acquire :} Si el cerrojo está libre, lo obtiene y lo marca como ocupado.
	\item \textbf{Realese :} Libera el cerrojo.
\end{itemize}
Cabe aclarar que nadie excepto el hilo que tiene adquirido el cerrojo puede liberarlo.\\

\begin{lstlisting}[style=C]
class Lock {
  public:
  // Constructor: inicia el cerrojo como libre
  Lock(const char* debugName);

  ~Lock();          // destructor
  const char* getName() { return name; }	// para depuracion

  // Operaciones sobre el cerrojo. Ambas deben ser *atomicas*
  void Acquire(); 
  void Release();

  // devuelve 'true' si el hilo actual es quien posee el cerrojo.
  // util para comprobaciones en el Release() y en las variables condicion
  bool isHeldByCurrentThread();	

  private:
    Thread *blocker; 	// thread que adquirio el cerrojo.
    const char* name;	// nombre del cerrojo.
    Semaphore *s;		// para depuracion
	Semaphore *semInvP; //Semaforo para control de prioridades
};
\end{lstlisting}
\subsection*{Implementación de cerrojo}
Para implementar los métodos, lo primero que se hizo es el método $Lock::isHeldByCurrentThread$
\begin{lstlisting}[style=C]
bool Lock::isHeldByCurrentThread(){
    return (blocker == currentThread);
}
\end{lstlisting}
Para implementar $Lock::Acquire$ se realizaron los siguientes pasos:
\begin{enumerate}
	\item Preguntar si el cerrojo esta libre, utilizando $isHeldByCurrentThread$.
	\item Decrementar el semáforo $s$.
	\item Asignar $currentThread$ a la variable $blocker$, para indicar que el thread $currentThread$ obtuvo el cerrojo.
\end{enumerate}
En cambio, para implementar $Lock::Release$ se realizaron los siguientes pasos:
\begin{enumerate}
	\item Hay que asegurar, mediante un ASSERT, que el thread actual es el que había obtenido el cerrojo. Para ello, se tiene que cumplir $isHeldByCurrentThread$.
	\item Incrementar el semáforo $s$.
	\item Liberar la variable $blocker$, para indicar que se liberó el cerrojo.
\end{enumerate}
\subsection*{Implementación de variables de condición}