\chapter{Resolución de la Práctica N\grad2: Sincronización de hilos en NachOS}
\section{Implementación de cerrojo y variables de condición}
\subsection*{Cerrojo}
\textsf{\underline{Marco teórico}}\\
Un cerrojo provee una exclusión mutua para utilizar ciertos recursos. Puede tener dos estados: libre y ocupado. Cabe aclarar que nadie excepto el hilo que tiene adquirido el cerrojo puede liberarlo. Además, sólo se permiten dos operaciones:
\begin{itemize}
	\item \textbf{Acquire :} Si el cerrojo está libre, lo obtiene y lo marca como ocupado.
	\item \textbf{Realese :} Libera el cerrojo.
\end{itemize}
\begin{lstlisting}[style=C]
class Lock {
  public:
	// Constructor: inicia el cerrojo como libre
  	Lock(const char* debugName);

  	~Lock();          // destructor
  	const char* getName() { return name; }	// para depuracion

  	// Operaciones sobre el cerrojo. Ambas deben ser *atomicas*
  	void Acquire(); 
  	void Release();

  	// devuelve 'true' si el hilo actual es quien posee
  	// el cerrojo. Util para comprobaciones en el Release()
  	// y en las variables condicion.
  	bool isHeldByCurrentThread();
  private:
    Thread *blocker; 	// thread que adquirio el cerrojo.
    const char* name;	// nombre del cerrojo.
    Semaphore *semLock;     // Semaforo para aislar la zona compartida
};
\end{lstlisting}
\textsf{\underline{Implementación}}\\
Para implementar los métodos, lo primero que se hizo es el método \texttt{Lock::isHeldByCurrentThread}
\begin{lstlisting}[style=C]
bool Lock::isHeldByCurrentThread(){
    return (blocker == currentThread);
}
\end{lstlisting}
En el constructor, se crea un semáforo \texttt{semLock} cuyo valor inicial es \texttt{1}.\\
Para implementar \texttt{Lock::Acquire} se realizaron los siguientes pasos:
\begin{enumerate}
	\item Preguntar si el cerrojo esta libre, utilizando \texttt{isHeldByCurrentThread}.
	\item Decremento el semáforo \texttt{semLock}.
	\item Asignar \texttt{currentThread} a la variable \texttt{blocker}, para indicar que dicho thread obtuvo el cerrojo.
\end{enumerate}
En cambio, para implementar \texttt{Lock::Release} se realizaron los siguientes pasos:
\begin{enumerate}
	\item Hay que asegurar, mediante un ASSERT, que el thread actual es el que había obtenido el cerrojo. Para ello, se tiene que cumplir \texttt{isHeldByCurrentThread}.
	\item Incrementar el semáforo \texttt{semLock}.
	\item Liberar la variable \texttt{blocker}, para indicar que se liberó el cerrojo.
\end{enumerate}
\subsection*{Variables de condición}
\textsf{\underline{Marco teórico}}\\
 Las variables de condición usadas en conjunto con cerrojo permiten a un hilo esperar por la ocurrencia de una condición arbitraria. Se utilizan para encolar hilos que esperan (\texttt{Wait}) a que otro hilo les avise (\texttt{Signal}) cuando se cumpla dicha condición.\\
Se definen tres operaciones sobre una variable condición:
\begin{itemize}
	\item \textbf{Wait: } Libera el cerrojo y expulsa al hilo de la CPU. El hilo se espera hasta que alguien le hace un \texttt{Signal}.
	\item \textbf{Signal: } Si hay alguien esperando en la variable, despierta a uno de los hilos. Si no hay nadie esperando, no ocurre nada.
	\item \textbf{Broadcast: } Despierta a todos los hilos que están esperando.
\end{itemize}
Todas las operaciones sobre una variable condición deben ser realizadas adquiriendo previamente el cerrojo. Esto significa que las operaciones sobre variables condición han de ejecutarse en exclusión mutua.
\newpage
\begin{lstlisting}[style=C]
class Condition {
  public:
    // Constructor: se le indica cual es el cerrojo al que pertenece
    // la variable condicion
	Condition(const char* debugName, Lock* conditionLock);	

    // libera el objeto
    ~Condition();	
    const char* getName() { return (name); };

    // Las tres operaciones sobre variables condicion.
    // El hilo que invoque a cualquiera de estas operaciones debe tener
    // adquirido el cerrojo correspondiente; de lo contrario se debe
    // producir un error.
    void Wait(); 	
    void Signal();   
    void Broadcast();

  private:
    const char* name; // Nombre de la v.c.
    Lock* lock;	// Cerrojo vinculado a la v.c.
    List<Semaphore*> * semList; // Lista de semaforos correspondiente a los
                                // hilos que se fueron bloqueando.
};
\end{lstlisting}
\textsf{\underline{Detalles de la implementación}}\\
En el constructor, se inicializa a \texttt{semList} con una lista vacía; se guardan el nombre del semáforo y el cerrojo vinculado, en las variables correspondientes.\\
Para implementar \texttt{Condition::Wait} se realizaron los siguientes pasos:
\begin{enumerate}
	\item Hay que asegurar, mediante un ASSERT, que el thread actual es el que había obtenido el cerrojo. Para ello, se tiene que cumplir \texttt{isHeldByCurrentThread}.
	\item Se crea un semáforo llamado \texttt{sem}, cuyo valor inicial es \texttt{0}.
	\item Se lo agrega al final de la lista \texttt{semList}.
	\item Se libera el \texttt{lock}.
	\item Se decrementa el semáforo \texttt{sem}, logrando que se bloquee el thread que esta ejecutándose (\texttt{currentThread}).
	\item Se obtiene el \texttt{lock}.
\end{enumerate}
Para implementar \texttt{Condition::Signal} se realizaron los siguientes pasos:
\begin{enumerate}
	\item Hay que asegurar, mediante un \texttt{ASSERT}, que el thread actual es el que había obtenido el cerrojo. Para ello, se tiene que cumplir \texttt{isHeldByCurrentThread}.
	\item Si hay algún semáforo en la lista \texttt{semList}, se quita el primero y se lo incrementa. De esta forma, logramos que se libere el thread bloqueado, para que se agregue a la lista de threads listos para ejecutarse.
\end{enumerate}
Finalmente, para implementar \texttt{Condition::Broadcast} se realizaron los siguientes pasos:
\begin{enumerate}
	\item Hay que asegurar, mediante un ASSERT, que el thread actual es el que había obtenido el cerrojo. Para ello, se tiene que cumplir \texttt{isHeldByCurrentThread}.
	\item Mientras haya algún semáforo en la lista \texttt{semList}, se invoca a \texttt{Signal}. De esta forma, logramos que se liberen todos los threads que se encontraban bloqueados.
\end{enumerate}
\section{Implementación de puertos}
\textsf{\underline{Introcción}}\\
La siguiente clase implementa un sistema de mensajes entre hilos, el cual permite que los emisores se
sincronicen con los receptores mediante un buffer.\\
Se definen dos operaciones: \textbf{\texttt{Send}} y \textbf{\texttt{Receive}}. La primera espera a que esté libre el buffer e inserta un mensaje en él. La segunda, se encarga de la recepción de un mensaje, bloqueándose mientras no llegue ninguno.
\begin{lstlisting}[style=C]
class Puerto {
    public:
        Puerto(const char* name);
        ~Puerto();
        const char* getName() { return portname; }	// para depuracion

        void Send(int message);
        void Receive(int* postbox);
    private:
	const char* portname; // nombre del puerto.
        Lock* lockPort; 	// cerrojo asociado a las v.c.
        Condition* sending; // v.c. que contralan el acceso
        Condition* receiving; // sincronizado al buffer.
        int  buffer; 	// buffer de comunicacion
        bool access;	// indica si el buffer esta libre.
}
\end{lstlisting}
\textsf{\underline{Detalles de la implementación}}\\
En el constructor, se crean el cerrojo y las variables de configuración. Además, se inicializa el nombre del puerto y se indica que el buffer está libre.\\
Para implementar \texttt{Puerto::Send}, se realizaron los siguientes pasos:
\begin{enumerate}
	\item Se obtiene el cerrojo \texttt{lockPort}.
	\item Mientras el buffer esté ocupado, se invoca al método \texttt{Condition::Wait}, correspondiente a la v.c $sending$.
	\item Se guarda el mensaje recibido en \texttt{buffer}.
	\item Se indica que el buffer está ocupado, seteando \texttt{access} en \texttt{false}.
	\item Se libera el cerrojo \texttt{lockPort}.
\end{enumerate}
Finalmente, para implementar \texttt{Puerto::Receive}, se realizaron los siguientes pasos:
\begin{enumerate}
	\item Se obtiene el cerrojo \texttt{lockPort}.
	\item Mientras el buffer esté libre, se invoca al método \texttt{Condition::Wait}, correspondiente a la v.c \texttt{receiving}.
	\item Se copia \texttt{buffer} en el argumento.
	\item Se indica que el buffer está libre, seteando \texttt{access} en \texttt{true}.
	\item Se libera el cerrojo \texttt{lockPort}.
\end{enumerate}
\section{Implementación del método $Thread::Join$}
El método \texttt{Thread::Join} permite al thread que la invoca esperar la finalización de otro proceso antes de continuar con su ejecución.\\
Para implementarlo, se agregó lo siguiente a la clase \texttt{Thread}:
\begin{lstlisting}[style=C]
class Thread {
public:
	...
	void Join(); //  Bloquea al llamante hasta que el hilo en cuestion termine.
private:
	int joinFlag; // Indica si se tiene que realizar el join.
	Puerto* joinPort; // Sirve para sincronizar los dos threads.
}
\end{lstlisting}
En el constructor, se inicializaron las variables agregadas; en el destructor, se libera el miembro \texttt{joinPort}.\\
Luego, se implementó el método \texttt{Thread::Join}:
\begin{lstlisting}
void
Thread::Join()
{
    if(joinFlag){
        int * msg = (int*) malloc (sizeof(int));
        joinPort->Receive(msg);
    }
}
\end{lstlisting}
Además, se modificaron los siguientes métodos:
\begin{itemize}
	\item \textbf{\texttt{Thread::Fork}} Se le agregó un argumento para setear la bandera \texttt{joinFlag} al momento de crear un hijo. Si el parámetro viene con un número mayor a cero, significa que el padre espera a que este nuevo hijo termine.
	\item \textbf{\texttt{Thread::Finish}} Si está activada la bandera \texttt{joinFlag}, se envía un mensaje a través del puerto \texttt{joinPort}, indicando de esta forma que finalicé. Por lo tanto, puedo desbloquear el otro thread.
\end{itemize}
Finalmente, mostraremos la secuencia de ejecución para utilizar esta característica:
\begin{enumerate}
    \item El thread \texttt{Padre} ejecuta unas tareas.
    \item \texttt{Padre} crea un thread \texttt{Hijo} indicando que quiere hacer \texttt{join}:
        \begin{lstlisting}
            Thread *hijo = new Thread("Hijo");
            hijo->Fork(childFunction, (*void) 1, 1); //Al ser el 3er argumento mayor que 0, indica que quiero hacer join.
         \end{lstlisting}
    \item El padre se queda esperando a que el hijo termine mediante:
        \begin{lstlisting}
            hijo->Join();
        \end{lstlisting}
    \item Cuando el thread \texttt{Hijo} termina, su padre continua su ejecución.
\end{enumerate}
\section{Implementación de multicolas con prioridad}
\subsection*{Implementación de la propiedad prioridad en los Threads}
A cada proceso se le determinará, mediante un criterio, su prioridad. Para ello, se agregó una propiedad a los Threads llamada \texttt{priority}, que indica cuál es la prioridad de dicho Thread.\\
Todos los threads tienen una prioridad positiva, siendo 0 la prioridad de menor importancia.\\
Se realizaron los siguientes cambios en el archivo \textbf{\texttt{thread.h}}:
\begin{lstlisting}
  public:
    ...
    int getPriority(){ return priority; } //Retorna la prioridad.
    void setPriority(int p){ priority = p; } // Asigna la prioridad.
  private:
    ...
    int priority; // Indica la prioridad.
\end{lstlisting}
Al constructor de \texttt{Threads} se le agregó un parámetro para indicar la prioridad de un thread.
\subsection*{Modificaciones al Scheduler para que utilice Multicolas}
\textsf{\underline{Esquema de planificación Múltiples colas de prioridad}}\\
Se definen la cantidad de colas que se encuentran en el sistema. Su número determina cuántas prioridades distintas se tienen para clasificar los procesos.\\
Cuando un proceso tenga su estado en listo, será colocado en la cola correspondiente a su prioridad. Una aclaración importante es que cada cola puede manejar un algoritmo de planificación diferente a las demás.\\
Se modificó el archivo \textbf{\texttt{scheduler.h}} de la siguiente manera:
\begin{lstlisting}
class Scheduler {
    public:
    ...
    
    private:
        List<Thread*> **readyList;  // Priority queue of threads
                                    // that are ready to run,
                                    // but not running
        int max_priority;    // Cantidad de colas.
}
\end{lstlisting}
\textsf{\underline{Implementación}}\\
Para desarrollar las colas de prioridad múltiple, se modificó \textbf{\texttt{scheduler.cc}}:
\begin{itemize}
    \item \textbf{Constructor de la clase} Se indica la cantidad de colas. Además, se inicializan las colas.
    \item \textbf{Destructor de la clase} Se eliminan las colas.
    \item \textbf{\texttt{Scheduler::ReadyToRun}} Agrega al thread al final de la cola correspondiente a su prioridad.
    \item \textbf{\texttt{Scheduler::FindNextToRun}} Empezando por las colas de mayor prioridad, obtiene un thread para ejecutarse. Si la cola superior está vacía, sigue con la cola de prioridad inmediatamente menor. Y así sucesivamente hasta encotrar un thread.
\end{itemize}
\subsection*{Inversión de prioridades}
\textsf{\underline{Descripción del problema}}\\
Sean dos procesos A,B que comparten un recurso. Supongamos que se da la siguiente situación:
\begin{itemize}
    \item El proceso A, de prioridad baja, pide el recurso compartido y es interrumpido luego de obtenerlo.
    \item El proceso B, de prioridad alta, pide el recurso compartido.
\end{itemize}
Esta situación hace que queden invertidas las prioridades relativas entre ambos, ya que el proceso B tiene que esperar a que el proceso A libere el recurso compartido. Como consecuencia, el proceso de mayor prioridad se ejecutará luego del proceso de menor prioridad.\\
\\\textsf{\underline{Detalles de la implementación de la solución}}\\
Para solucionar el problema de inversión de prioridades, aplicamos una técnica llamada \emph{Herencia de Prioridades}: dados dos procesos que comparten un recurso, tales que tengan distintas prioridades. Supongamos que el proceso de menor prioridad esté bloqueando el recurso compartido. Entonces procedemos a cambiar la prioridad de dicho proceso, asignándole la prioridad del otro.\\
En NachOS, este problema puede ocurrir cuando dos threads comparten un cerrojo o una variable de condición. Para solucionar este problema, se modificaron las siguientes clases:
\begin{itemize}
    \item \texttt{synch.h} y \texttt{synch.cc}: Se agregó un cerrojo llamado \texttt{invPrController}, cuya función es controlar el cambio de prioridades.
    \item \texttt{scheduler.h} y \texttt{scheduler.cc}: Se agregó un método llamado \texttt{Scheduller::ChangePriorityQueue} cuya función es cambiar de cola a un thread. Para ello, recibe dos argumentos: un thread y una prioridad.
    \item \texttt{Lock::Acquire}: Se modificó este método de modo tal que si un thread \texttt{T2} quiere adquirir el cerrojo pero éste está en manos de otro thread \texttt{T1} cuya prioridad es menor que la de \texttt{T2}, se proceda a asignarle a \texttt{T1} la misma prioridad que \texttt{T2}. Para ello, se invoca al método \texttt{Scheduller::ChangePriorityQueue}.
\end{itemize}
¿Por qué no se puede aplicar herencia de prioridades en semáforos?\\
Notemos que la clase \texttt{Semaphore} guarda una cola con los threads bloqueados pero se olvida del thread que lo bloquea. Por lo tanto, no sabemos a cuál thread cambiar la prioridad.