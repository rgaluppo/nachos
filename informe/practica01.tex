\chapter{Resolución de la Práctica N\grad1}
\begin{itemize}
    \item \textit{¿Cuánta memoria tiene la máquina simulada para \textbf{\textit{NachOS}}?}\\
    La máquina simulada tiene 32 páginas de 128 bytes.
    \item \textit{¿Cómo cambiaría ese valor?}\\
    Para cambiar ese valor hay que modificar la cantidad de páginas y/o modificar el tamaño de cada una de ellas. Las respectivas variables son \texttt{NumPhysPages} y \texttt{PageSize}. 
    \item \textit{¿De qué tamaño es un disco?}\\
    El disco simulado por \textbf{\textit{NachOS}} tiene 131 MB. Su estructura interna consta de 32 pistas de 32 sectores de 128 bytes cada una.
    \item \textit{¿Cuántas instrucciones del \textbf{\textit{MIPS}} simula \textbf{\textit{NachOS}}?}\\
\textbf{\textit{NachOS}} puede simular hasta 63 instrucciones del \textbf{\textit{MIPS}} (\texttt{MaxOpcode=63}) aunque sólo hay definidas 59 de ellas. El listado se encuentra en \texttt{mipssim.h}.
    \item \textit{Explicar el código que procesa la instrucción \texttt{add}}\\
\begin{lstlisting}[style=C]
case OP_ADD:
  sum = registers[(int)instr->rs] + registers[(int)instr->rt];
  if (!((registers[(int)instr->rs] ^ registers[(int)instr->rt])
             &  SIGN_BIT) 
      && ( (registers[(int)instr->rs] ^ sum)
             & SIGN_BIT)) {

                   RaiseException(OverflowException, 0);
                   return;
    }
    registers[(int)instr->rd] = sum;
\end{lstlisting}

   
\end{itemize}

    

